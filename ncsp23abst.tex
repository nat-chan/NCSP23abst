\documentclass[a4paper]{article}
\usepackage[colorlinks, urlcolor=blue,
	pdffitwindow, bookmarksopen,
	filecolor=blue,
	bookmarksopenlevel=0,
	pdfpagelayout=SinglePage, dvipdfm]{hyperref}
\usepackage{ncsp20} %% !! Strongly recommended to use it
\usepackage{color}
\usepackage{multicol}
\usepackage{geometry}
\usepackage{here}
\usepackage{caption}
\usepackage{indentfirst}
\usepackage[dvipdfmx]{graphicx}
\usepackage[export]{adjustbox}
%\usepackage{showframe}
\usepackage{amsmath}
\geometry{left=18mm,right=18mm,top=0mm,bottom=10mm}

% {{{
% 2段組み環境内でfigure環境
\newenvironment{Figure}
  {\par\medskip\noindent\minipage{\linewidth}}
  {\endminipage\par\medskip}
  
%セクションの前後の空白を詰める
\newcommand{\aftersection}{\vspace{-5pt}}
\newcommand{\beforesection}{\vspace{-10pt}}
\newcommand{\beforecaption}{\vspace{-10pt}}

% }}}

\begin{document}
\name{} \address{} %\titleがncsp20でrenewcommandされているので空文字にする
\title{Aesthetic and high-following line art colorization}
%\title{Aesthetic and robust colorization for rough sketches}
\maketitle
\vspace{-80pt} % XXX タイトルと本文の間を詰める

\begin{center}
    \begin{tabular}[t]{c}
    {\large Natsuki Ogino}\footnotemark[1]
    \and 
    {\large Kazumasa Horie}\footnotemark[2]
    \and 
    {\large Amagasa Toshiyuki}\footnotemark[2]
    \end{tabular}
\end{center}
\footnotetext[1]{
Graduate School of Science and Technology Degree Programs, University of Tsukuba, 1-1-1 Tennodai, Tsukuba, Ibaraki, 305-8577, Japan.\\
\hspace{4.5mm} E-mail: ogino.natsuki.tm@alumni.tsukuba.ac.jp}
\footnotetext[2]{
Center for Computational Sciences, University of Tsukuba, 1-1-1 Tennodai, Tsukuba, Ibaraki, 305-8577, Japan.\\
\hspace{4.5mm} Phone: 029-855-5385
E-mail: \{horie, amagasa\}@cs.tsukuba.ac.jp}

\newcommand{\mypt}{12.1pt}
\newcommand{\mythh}[2]{ \scalebox{0.5}{\begin{tabular}{c} {\fontsize{\mypt}{\mypt}\selectfont #1} \\ {\fontsize{\mypt}{\mypt}\selectfont #2} \end{tabular}} }
\newcommand{\mythhh}[3]{ \scalebox{0.5}{\begin{tabular}{c} {\fontsize{\mypt}{\mypt}\selectfont #1} \\ {\fontsize{\mypt}{\mypt}\selectfont #2} \\ {\fontsize{\mypt}{\mypt}\selectfont #3} \end{tabular}} }
\newcommand{\mytd}[1]{ \includegraphics[height=.05\textheight,valign=b]{#1}}

\vspace{-10pt}
\begin{table}[hbtp]
\begin{tabular}{@{}c@{}|@{}c@{}|@{}c@{}|@{}c@{}|@{}c@{}|@{}c@{}|@{}c@{}|@{}c@{}|@{}c@{}|@{}c@{}|@{}c@{}}
\mythh{Reference image}{for colorization}&
\mythhh{Line drawing}{extraction source}{image}&
\mythh{Line drawing}{image input}&
\mythhh{Ours}{DF-projection}{w/ref}&
\mythhh{Baseline}{pixel2style2pixel}{w/ref}&
\mythh{style2paints v4.5}{w/ref}&
\mythhh{Pixiv(PFN)}{Petalica Paint}{Tanpopo wo/ref}&
\mythhh{Pixiv(PFN)}{Petalica Paint}{Satsuki wo/ref }&
\mythhh{Pixiv(PFN)}{Petalica Paint}{Canna wo/ref}&
\mythh{NAVER Webtoon}{AI Painter wo/ref}&
\mythhh{Celsys}{ClipStudioPaint}{v1.12.8 wo/ref}
\\ \hline
\mytd{examples/synth/seed2117.png} & \mytd{examples/synth/seed0060.png} & \mytd{examples/sim/seed0060.png} & \mytd{grid100ref/dist_weight=0.3_additional_weight=0.7_custom_lr=0.05_w_std=0.7/3.png}  & \mytd{examples/sim_synth/60_2117.png}   & \mytd{examples/pair50/3/a.png}  & \mytd{examples/petalica/t3.jpg}  & \mytd{examples/petalica/s3.jpg}  & \mytd{examples/petalica/c3.jpg}  & \mytd{examples/webtoon/3_ai-painter.png}  & \mytd{examples/clipstudio/3.png}  \\ \hline
\mytd{examples/synth/seed2133.png} & \mytd{examples/synth/seed0092.png} & \mytd{examples/sim/seed0092.png} & \mytd{grid100ref/dist_weight=0.3_additional_weight=0.7_custom_lr=0.05_w_std=0.7/5.png}  & \mytd{examples/sim_synth/92_2133.png}   & \mytd{examples/pair50/5/a.png}  & \mytd{examples/petalica/t5.jpg}  & \mytd{examples/petalica/s5.jpg}  & \mytd{examples/petalica/c5.jpg}  & \mytd{examples/webtoon/5_ai-painter.png}  & \mytd{examples/clipstudio/5.png}  \\ \hline
\mytd{examples/synth/seed3337.png} & \mytd{examples/synth/seed2061.png} & \mytd{examples/sim/seed2061.png} & \mytd{grid100ref/dist_weight=0.3_additional_weight=0.7_custom_lr=0.05_w_std=0.7/48.png} & \mytd{examples/sim_synth/2061_3337.png} & \mytd{examples/pair50/48/a.png} & \mytd{examples/petalica/t48.jpg} & \mytd{examples/petalica/s48.jpg} & \mytd{examples/petalica/c48.jpg} & \mytd{examples/webtoon/48_ai-painter.png} & \mytd{examples/clipstudio/48.png} 
\end{tabular}
\beforecaption
\captionof{figure}{Colorization results of proposed, baseline, and comparative methods}
\label{tab:compare}
\end{table}


\vspace{-20pt}
\begin{multicols}{2}
\section*{Objective}
\aftersection

Technology for automatically applying aesthetic coloring to line drawings is in high demand for applications that support illustrators' work. 
%However, the results of previous methods are often flat and are far from beautifully finished illustrations. They are also not robust to rough line drawings.
Recently, a new general-purpose img2img work frame using StyleGAN as a decoder, called pixel2style2pixel(pSp), has been proposed. Using this method for this task solved the above problems. However, we found that the method lost the ability to follow the input line drawing. There is a trade-off between the aesthetic quality of the coloring result and the ability to follow the input line drawing. This research aims to improve this trade-off.

\beforesection
\section*{Methods}
\aftersection
\vspace{-15pt}

\small
\begin{align*}
\frac{1}{HW}\sum_{h,w}\left({\rm DF \circ S\circ G}(w^+_{start}+w_{opt})-{\rm DF \circ S}(I_{target})\right)^2
\end{align*}
\normalsize

In this study, we introduced a new metric, DF-MSE, which is expressed in the above equation, as a measure of the followability to the input line art. It measures the degree of matching between the line drawing extracted from the generated image and the input line drawing. In order to adapt to the diversity of sketches created by users, the model Simplify(S) is used to normalize line thickness and density and then converted to a Distance Field, and pixel-wise MSE is calculated. Conventional studies of line drawing coloring use SSIM and PSNR indices to measure the fidelity of the coloring process, using both human-colored and model-colored images. However, using these indices to measure line drawing fidelity causes problems, such as sensitivity to visually unrecognizable one-pixel deviations and local scaling. Therefore, this problem is solved by converting each pixel in the image into a distance to the nearest line.

We used the Aesthetic score predictor to evaluate and compare the aesthetics of the coloring results. We conducted an AB test of the generated images' aesthetics to verify this metric's accuracy. It was confirmed that the Aesthetic score predictor has the same level of sensitivity to aesthetics as a human subject with average sensitivity. Additionally, this metric has been used in prestigious publications such as stable-diffusion.

The architecture of the proposed method is described in detail. We propose a new architecture, DF-projection, which uses the $w^+$ vector obtained from the encoder of the baseline method pixel2style2pixel as the initial point and the DF-MSE mentioned above as the weighted loss to project into the StyleGAN latent space.


%\tiny
%\begin{align*}
%\mathcal{L}_{{\rm DF}}&=\frac{1}{HW}\sum_{h,w}\left({\rm DF \circ S\circ \underset{synth}{G}}(w^+_{start}+w_{opt})-{\rm DF \circ S}(I_{target})\right)^2
%\\
%\mathcal{L}_{{\rm VGG}}&=\sum_{feature\ wise}\left({\rm VGG \circ \underset{synth}{G}}(w^+_{start}+w_{opt})-{\rm VGG}(I_{target})\right)^2
%\\
%\mathcal{L}_{{\rm project}} &= \lambda_{{\rm DF}} \mathcal{L}_{{\rm DF}} + (1-\lambda_{{\rm DF}})\mathcal{L}_{{\rm VGG}}+\lambda_{{\rm reg}}\mathcal{L}_{{\rm reg}}
%\end{align*}
%\normalsize


\beforesection
\section*{Results}
\aftersection

To evaluate the performance of the proposed method, we conducted experiments comparing it with the baseline method, the proposed method, and a coloring method already in use in practice. We prepared 80,000 training, 2,000 validation, and 2,000 test dataset by the following procedure: removing images using a tag that was not appropriate for research, whitening the background, aligning faces, and extracting line drawings from the Danbooru2020 dataset. Since obtaining permission to publish these images in a paper was difficult, We prepared 100 pairs of images generated by StyleGAN in this experiment as supplemental datasets. In addition, we released the method as a demo application and collected 100 line drawings created by users. Some of these images were accepted for publication.

\begin{Figure}
\scalebox{0.6}{
\begin{tabular}{c|c|c|c|c|c|c|c|c}
                   & \textbf{Ours}   & pSp    & s2p   & Tanpopo & Satsuki & Canna & webtoon & clipstudio \\
\hline
\textbf{Aesthetic WP↑} & 38.00  & 100-x  & 24.00 & 4.00    & 10.00   & 6.00  & 6.00    & 6.00       \\
% " & ".join([f"{float(t.strip())**.5:.2f}" for t in txt.replace("&", "\n").split("\n")])
%DF-MSE↓            & 148.05 & 183.91 & 19.72 & 31.87   & 3.12    & 49.00 & 17.34   & 7.31       \\
\textbf{DF-RMSE↓} & 12.17 & 13.56 & 4.44 & 5.65 & 1.77 & 7.00 & 4.16 & 2.70 \\
\hline
SSIM↑              & 0.49   & 0.50   & 0.70  & 0.73    & 0.67    & 0.65  & 0.65    & 0.68       \\
PSNR↑              & 11.43  & 11.59  & 12.73 & 14.01   & 11.70   & 11.74 & 11.53   & 11.80      \\
\end{tabular}
}
\beforecaption
\captionof{table}{Quantitative evaluation results}
\label{tab:metrics}
\end{Figure}
Quantitative evaluation results showed that the baseline method significantly improved the aesthetic score compared to previous methods. However, on the other hand, the tracking of the input line drawings became worse. The proposed method improves the followability of the input line art with only slightly deteriorating the aesthetic score compared to the baseline method.

\beforesection
\section*{Conclusions}
\aftersection

This research tackled line art colorization tasks and improved the follow-up to the input line drawing while maintaining the aesthetic quality of the coloring result.
In the camera-ready paper, we describe a quantitative evaluation of a test dataset. We also show that our method is effective in coloring on-the-wild line drawings created by the user, with high followability to the input line drawing and aesthetic line drawing coloring. We also show that the method can manipulate attributes such as gender and age when coloring line drawings by utilizing the latent space image editing feature of StyleGAN.

% Reference
%\begin{thebibliography}{9}
%\bibitem{psp}
%Richardson, et al., "Encoding in Style: a StyleGAN Encoder for Image-to-Image Translation", Proceedings of Computer Vision and Pattern Recognition, 2021. 
%\end{thebibliography}

%methodsがながい
%resultsの表上を半分にして考察を伸ばす(具体的な例をあげる)
%objectiveがつぶすとしたら前半ちょっと軽めにする

\end{multicols}\end{document}
